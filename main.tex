% !TEX program = xelatex

%% main.tex
%% Copyright 2025 Guodong Li
%
% This work may be distributed and/or modified under the
% conditions of the LaTeX Project Public License, either version 1.3
% of this license or (at your option) any later version.
% The latest version of this license is in
%   https://www.latex-project.org/lppl.txt
% and version 1.3c or later is part of all distributions of LaTeX
% version 2008 or later.
%
% This work has the LPPL maintenance status `maintained'.
% 
% The Current Maintainer of this work is Guodong Li.
%
% This work consists of the files sdu-thesis.cls main.tex

\documentclass{settings/sdu-thesis}



\begin{document}

% 论文信息
% !TEX root = ../main.tex

% info.tex
% 中图分类号
\fenlei{TP309}

% 单位代号
\DWdaihao{10422}
% 密级
\miji{公\quad 开}
% 学号
\StuNum{123456789}
% 页眉
\ThesisHeader{山东大学博士学位论文}
\EThesisHeader{Dissertation for Doctoral Degree}
% 论文类型(封面) 中
\CThesisType{博士学位论文}
% 论文类型(封面) 英
\EThesisType{Dissertation for Doctoral Degree}
% 中文标题
\Ctitle{山东大学博士学位论文\LaTeX{}模板}
% 英文标题
\Etitle{A \LaTeX{} Template for Doctoral Dissertation of Shandong University}
% 作者中文名
\Cauthor{哈利波特}
% 学院
\Dpart{格兰芬多学院}
% 专业中文名
\Cmajor{魔法专业}
% 年级
\Grade{}
% 指导老师中文名
\Csuperver{笨笨虎钓鱼}
\Ccosuperver{}
% 中文日期
\Cdate{2025年5月29日}
%\Cdate{\today}
\title{\the\Ctitle}
\author{\the\Cauthor}
\date{\the\Cdate}

% 学位类型
\DegreeType{}        %学术学位 不填此项
%\DegreeType{(专业学位)} %专业学位

\maketitlepagestatement

% !TEX root = ../main.tex

\begin{cnabstract}

    随着大数据与人工智能技术的飞速发展,跨机构的数据融合与协同计算已成为释放数据价值的关键途径。然而,数据孤岛现象与日益严格的隐私保护法律法规之间的矛盾,使得如何在保护数据隐私的前提下实现数据价值的安全流通成为亟待解决的问题。作为安全多方计算(Secure Multi-party Computation, MPC)的重要分支,隐私集合运算(Private Set Operations, PSO)允许参与方在不泄露私有数据的前提下协同计算集合的交集、并集等信息。尽管两方场景下的技术已趋于成熟,但多方场景(Multi-Party Private Set Operations, MPSO)在安全性、实用性、功能性和统一性方面都存在严重不足,仍面临严峻挑战:一方面,现有的多方隐私集合求并(Multi-party Private Set Union, MPSU)协议或依赖不切实际的“非合谋假设”,难以抵御现实世界存在的任意共谋攻击,或在计算与通信复杂度上未能达到线性,性能难以满足实际应用的需求;另一方面,现有 MPSO 协议功能单一,难以
满足现实场景中的复杂需求,且技术异构,开发部署维护成本高。目前学术界缺乏能够支持任意集合公式计算的 MPSO 统一框架。

    针对上述挑战,本文深入研究了 MPSO 的核心理论与关键技术,从具体协议的突破到通用框架的构建,取得了一系列创新性成果:

    首先,针对 MPSU 协议安全性弱与效率低下的痛点,本文提出了一种新的密码学组件 —— 批量秘密分享隐私成员测试(batch secret-shared private membership test, batch ssPMT)协议,并以此为基础分别基于对称密钥技术和公钥密码技术提出了两种安全高效的MPSU 协议。其中在对称技术路线中,本文构造了首个在标准半诚实模型下证明安全的基于对称密钥的 MPSU 协议。该协议不仅成功消除了之前 的 SOTA 协议 \cite{LG-ASIACRYPT-2023} 对非共谋假设的依赖,显著增强了安全性,还表现出更加优异的具体性能。在局域网(LAN)环境下,其在线阶段的运行效率提升了 $3.9 \sim 10.0$ 倍,整体运行效率提升了 $1.2 \sim 7.8$ 倍。在公钥技术路线中,本文构造了首个同时实现线性计算复杂度和线性通信复杂度的 MPSU 协议,其总通信量相比于之前 SOTA 协议 \cite{LG-ASIACRYPT-2023} 降低了 $3.0 \sim 36.5$ 倍,在带宽受限的广域网(WAN)环境下具有显著优势。

    其次,为了解决现有 MPSO 协议功能局限以及缺乏统一性的难题,本文通过引入新的密码学原语 —— 谓词零分享(Predicative Zero-Sharing),基于对称密钥技术构建了首个实用的 MPSO 统一框架。该框架不仅能够计算由交、并、差运算任意组合构成的集合公式,还可扩展至支持更为复杂的任意集合公式结果求势(MPSO-card)及基于电路的通用 MPSO(Circuit-MPSO)功能。

    最后,基于该 MPSO 框架,本文实例化了一系列安全高效的具体协议,填补了 MPSO 各个子领域的多项研究空白。其中,实例化的多方隐私集合求交(Multi-party Private Set Intersection, MPSI)协议是首个在标准半诚实模型下实现最优渐进复杂度(与明文传输方案相同量级)的基于对称密钥的 MPSI 方案,同时也是目前在线效率最高的 MPSI 协议,在 LAN 环境下比之前 SOTA 协议 \cite{WuYC24} 快了 $2.4 \sim 5.2$ 倍;实例化的多方隐私交集求势(MPSI-card)协议和多方隐私交集求势与和(MPSI-card-sum)协议是首个在标准半诚实模型下实现最优渐进复杂度的同类方案,其中 MPSI-card 协议具有目前同类协议中在线阶段的最佳性能, 其在线通信量比 SOTA \cite{ChenDGB22} 降低了 $14.0 \sim 20.3$ 倍,而 MPSI-card 协议是唯一拥有具体实现的同类方案;实例化的基于电路的 MPSI 协议(Circuit-MPSI)协议是首个在不诚实大多数设定下安全的同类方案,突破了 Circuit-MPSI 仅限于诚实大多数安全的局限;实例化的 MPSU 协议是标准半诚实模型下基于对称密钥的又一高效 MPSU 方案,在理论上实现了更优的渐进复杂度;实例化的多方隐私并集求势协议(MPSU-card)协议和基于电路的 MPSU 协议(Circuit-MPSU)是目前唯一可用的同类构造。
    此外,本文还探索了基于公钥体制的 MPSO 框架构造,完善了 MPSO 的理论技术体系。

    \cnkeywords{安全多方计算;隐私集合运算;多方隐私集合运算;多方隐私集合求交;多方隐私集合求并}

\end{cnabstract}

\begin{enabstract}

    This document introduces a \LaTeX{} template for writing doctoral dissertations
    at Shandong University. The template is designed to help doctoral students
    write and format their dissertations quickly and efficiently according to the
    school's requirements. The template includes format settings for the cover,
    abstract, table of contents, main text, references, etc., and provides detailed
    instructions and sample code. By using this template, users can focus on
    writing the content of the dissertation without worrying about formatting
    issues, thereby improving the efficiency and quality of dissertation writing.

    This template is based on the sduthesis.cls template
    (\url{https://github.com/Liam0205/sduthesis/}), and I would like to thank the
    original author for his hard work. This template modifies some formats based on
    the referenced template, eliminates some warnings about fonts, and allows it to
    compile normally in my environment. However, due to my limited technical level,
    the template inevitably has some problems and shortcomings. I hope that users
    can criticize and correct them to improve them together.

    \enkeywords{Shandong University; Doctoral Thesis; \LaTeX{} Template}

\end{enabstract}

% 中文目录 和 英文目录
\content

% 插图目录 和 表格目录
\ftcontent

\chapter{模板使用示例}
\echapter{Template Usage Example}
\section{模板下载}
\esection{Template Download}
你可以从以下网址下载本模板:\url{https://github.com/gudngli/SDU-Thesis}。

\section{基本结构}
\esection{Basic Structure}
本模板的基本结构如下:
\begin{itemize}
    \item settings/:模板设置文件夹,包含论文的格式设置。
    \item main.tex:主文件,包含论文的整体结构。
    \item chapters/:章节文件夹,包含各个章节的内容。
    \item ref.bib:参考文献文件,包含所有引用的文献信息。使用cite命令引用参考文献,例如\cite{MSR-TIT24}。
\end{itemize}

\section{编译方法}
\esection{Compilation Method}
使用以下命令编译论文:
    \begin{verbatim}
        xelatex main.tex
        bibtex main.aux
        xelatex main.tex
        xelatex main.tex
    \end{verbatim}
或者
\begin{itemize}
    \item Windows:使用 TeX Live 或 MiKTeX 编译。
    \item MacOS:使用 MacTeX 编译。(或者终端运行``make''命令)
\end{itemize}
本文档已经在 TeXlive 2023 上测试通过。

\section{选项说明}
\esection{Options Description}

\begin{itemize}
    \item \texttt{[double]}: 双面打印。此时所有章节的首页都在奇数页,另外,页面边距也会有所调整,内侧边距比外侧边距大。以便装订。
    \item \texttt{[fzfont]}: 使用方正字体。默认使用的是宋体,如需在论文首页使用方正字体,请添加此选项。使用前确保自己的电脑上已经安装了``方正粗黑宋简体''字体。
    该字体的安装文件可以在settings文件夹下找到。
    \item \texttt{[print]}: 使用该选项时所有链接的颜色都会被设置为黑色。
\end{itemize}

\section{论文标题,作者信息设置}
\esection{Thesis Title, Author Information Setting}
在 \texttt{./chapters/info.tex} 文件中,你可以设置论文的标题,作者信息等。

\section{插入图片}
\esection{Insert Picture}
插入图片的示例代码如下:
\begin{verbatim}
    \begin{figure}[h]
        \centering
        \includegraphics[width=0.5\linewidth]{settings/logo.pdf}
        \caption[SDUlogo]{山东大学校徽}
        \label{fig:sdulogo}
    \end{figure}
\end{verbatim}
效果如下,请留意插图目录的变化:
\begin{figure}[h]
    \centering
    \includegraphics[width=0.5\linewidth]{settings/logo.pdf}
    \caption[SDUlogo]{山东大学校徽}
    \label{fig:sdulogo}
\end{figure}

\section{插入表格}
\esection{Insert Table}
插入表格的示例代码如下:
\begin{verbatim}
    \begin{table}[h]
        \centering
        \label{tab:sdulogo}
        \begin{tabular}{|c|c|}
            \hline
            山东大学 & 校徽 \\
            \hline
            山东大学 & 校徽 \\
            \hline
        \end{tabular}
        \caption{山东大学校徽}
    \end{table}
\end{verbatim}
效果如下,请留意表格目录的变化:
\begin{table}[h]
    \centering
    \label{tab:sdulogo}
    \begin{tabular}{|c|c|}
        \hline
        山东大学 & 校徽 \\
        \hline
        山东大学 & 校徽 \\
        \hline
    \end{tabular}
    \caption{山东大学校徽}
\end{table}

一个稍微复杂的表格示例:
\begin{table}[ht]
    \centering
    \renewcommand{\arraystretch}{1}
    \begin{tabular}{|l|l|c|}
        \hline
        \multicolumn{3}{|c|}{Optimal-access}                                                                       \\
        \hline
        \hline
        Codes                      & Sub-packetization $\ell$                                    & Restrictions    \\
        \hline
        \cite[Section VIII]{Ye16}  & $s^{n-1}$                                                   &                 \\
        \hline
        \cite{CB}                  & $s^{n}$                                                     &                 \\
        \hline
        \cite{Liu22}               & $s^{\left\lceil n/2 \right\rceil}$                          &                 \\
        \hline
        \cite{Vajha21}             & $s^{\left\lceil n/s \right\rceil}$      (smallest possible) & $s\in\{2,3,4\}$ \\
        \hline
        %Section~\ref{sect:code1}   & $s^{\left\lceil n/s \right\rceil}$      (smallest possible) &                 \\
        %\hline
        \multicolumn{3}{c}{}                                                                                       \\
        \hline
        \multicolumn{3}{|c|}{Not optimal-access}                                                                   \\
        \hline
        \hline
        Codes                      & Sub-packetization $\ell$                                    & Restrictions    \\
        \hline
        \cite[Section IV]{Ye16}    & $s^{n}$                                                     &                 \\
        \hline
        \cite{LWHY22}              & $s^{\left\lceil n/3 \right\rceil}$      (best-known)        & $d=k+1$         \\
        \hline
        \cite{Zhang23} \cite{Li23} & $s^{\left\lceil n/2 \right\rceil}$                          &                 \\
        \hline
        %Section~\ref{sect:code2}   & $s^{\left\lceil n/(s+1) \right\rceil}$    (best-known)      &                 \\
        %\hline
        %\multicolumn{3}{c}{}
    \end{tabular}
    \caption[MSR码的显式构造]{Explicit $(n,k)$ MSR code constructions with linear field size and $d<n-1$ helper nodes, where $s=d-k+1$.
        ``smallest possible'' means that $\ell$ achieves the theoretical lower bound; ``best-known'' means that $\ell$ is the smallest among all the existing constructions.}
    \label{tab:cmp}
\end{table}

% 参考文献
\references
\bibliographystyle{ieeetr}
\bibliography{ref}

% 致谢
\input{chapters/acknowledgment}

% 博士期间所发论文
\input{chapters/publications}

% 博士期间所获奖项
\input{chapters/awards}

% 学位论文评阅及答辩情况表
%\comments
%\begin{figure*}[htp]
%	\centering
%	\includegraphics[scale=0.82]{./contents/table.eps}
%\end{figure*}

\end{document}
