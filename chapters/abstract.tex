% !TEX root = ../main.tex

\begin{cnabstract}

    随着大数据与人工智能技术的飞速发展,跨机构的数据融合与协同计算已成为释放数据价值的关键途径。然而,数据孤岛现象与日益严格的隐私保护法律法规之间的矛盾,使得如何在保护数据隐私的前提下实现数据价值的安全流通成为亟待解决的问题。作为安全多方计算(Secure Multi-party Computation, MPC)的重要分支,隐私集合运算(Private Set Operations, PSO)允许参与方在不泄露私有数据的前提下协同计算集合的交集、并集等信息。尽管两方场景下的技术已趋于成熟,但多方场景(Multi-Party Private Set Operations, MPSO)在安全性、实用性、功能性和统一性方面都存在严重不足,仍面临严峻挑战:一方面,现有的多方隐私集合求并(Multi-party Private Set Union, MPSU)协议或依赖不切实际的“非合谋假设”,难以抵御现实世界存在的任意共谋攻击,或在计算与通信复杂度上未能达到线性,性能难以满足实际应用的需求;另一方面,现有 MPSO 协议功能单一,难以
满足现实场景中的复杂需求,且技术异构,开发部署维护成本高。目前学术界缺乏能够支持任意集合公式计算的 MPSO 统一框架。

    针对上述挑战,本文深入研究了 MPSO 的核心理论与关键技术,从具体协议的突破到通用框架的构建,取得了一系列创新性成果:

    首先,针对 MPSU 协议安全性弱与效率低下的痛点,本文提出了一种新的密码学组件 —— 批量秘密分享隐私成员测试(batch secret-shared private membership test, batch ssPMT)协议,并以此为基础分别基于对称密钥技术和公钥密码技术提出了两种安全高效的MPSU 协议。其中在对称技术路线中,本文构造了首个在标准半诚实模型下证明安全的基于对称密钥的 MPSU 协议。该协议不仅成功消除了之前 的 SOTA 协议 \cite{LG-ASIACRYPT-2023} 对非共谋假设的依赖,显著增强了安全性,还表现出更加优异的具体性能。在局域网(LAN)环境下,其在线阶段的运行效率提升了 $3.9 \sim 10.0$ 倍,整体运行效率提升了 $1.2 \sim 7.8$ 倍。在公钥技术路线中,本文构造了首个同时实现线性计算复杂度和线性通信复杂度的 MPSU 协议,其总通信量相比于之前 SOTA 协议 \cite{LG-ASIACRYPT-2023} 降低了 $3.0 \sim 36.5$ 倍,在带宽受限的广域网(WAN)环境下具有显著优势。

    其次,为了解决现有 MPSO 协议功能局限以及缺乏统一性的难题,本文通过引入新的密码学原语 —— 谓词零分享(Predicative Zero-Sharing),基于对称密钥技术构建了首个实用的 MPSO 统一框架。该框架不仅能够计算由交、并、差运算任意组合构成的集合公式,还可扩展至支持更为复杂的任意集合公式结果求势(MPSO-card)及基于电路的通用 MPSO(Circuit-MPSO)功能。

    最后,基于该 MPSO 框架,本文实例化了一系列安全高效的具体协议,填补了 MPSO 各个子领域的多项研究空白。其中,实例化的多方隐私集合求交(Multi-party Private Set Intersection, MPSI)协议是首个在标准半诚实模型下实现最优渐进复杂度(与明文传输方案相同量级)的基于对称密钥的 MPSI 方案,同时也是目前在线效率最高的 MPSI 协议,在 LAN 环境下比之前 SOTA 协议 \cite{WuYC24} 快了 $2.4 \sim 5.2$ 倍;实例化的多方隐私交集求势(MPSI-card)协议和多方隐私交集求势与和(MPSI-card-sum)协议是首个在标准半诚实模型下实现最优渐进复杂度的同类方案,其中 MPSI-card 协议具有目前同类协议中在线阶段的最佳性能, 其在线通信量比 SOTA \cite{ChenDGB22} 降低了 $14.0 \sim 20.3$ 倍,而 MPSI-card 协议是唯一拥有具体实现的同类方案;实例化的基于电路的 MPSI 协议(Circuit-MPSI)协议是首个在不诚实大多数设定下安全的同类方案,突破了 Circuit-MPSI 仅限于诚实大多数安全的局限;实例化的 MPSU 协议是标准半诚实模型下基于对称密钥的又一高效 MPSU 方案,在理论上实现了更优的渐进复杂度;实例化的多方隐私并集求势协议(MPSU-card)协议和基于电路的 MPSU 协议(Circuit-MPSU)是目前唯一可用的同类构造。
    此外,本文还探索了基于公钥体制的 MPSO 框架构造,完善了 MPSO 的理论技术体系。

    \cnkeywords{安全多方计算;隐私集合运算;多方隐私集合运算;多方隐私集合求交;多方隐私集合求并}

\end{cnabstract}

\begin{enabstract}

    This document introduces a \LaTeX{} template for writing doctoral dissertations
    at Shandong University. The template is designed to help doctoral students
    write and format their dissertations quickly and efficiently according to the
    school's requirements. The template includes format settings for the cover,
    abstract, table of contents, main text, references, etc., and provides detailed
    instructions and sample code. By using this template, users can focus on
    writing the content of the dissertation without worrying about formatting
    issues, thereby improving the efficiency and quality of dissertation writing.

    This template is based on the sduthesis.cls template
    (\url{https://github.com/Liam0205/sduthesis/}), and I would like to thank the
    original author for his hard work. This template modifies some formats based on
    the referenced template, eliminates some warnings about fonts, and allows it to
    compile normally in my environment. However, due to my limited technical level,
    the template inevitably has some problems and shortcomings. I hope that users
    can criticize and correct them to improve them together.

    \enkeywords{Shandong University; Doctoral Thesis; \LaTeX{} Template}

\end{enabstract}