\chapter{基于对称密钥的高效多方隐私集合求并协议}
\echapter{Efficient Multi-Party Private Set Union Protocol Based on Symmetric-Key Operations}

\section{引言}
\esection{Introduction}

隐私集合求并(Private Set Union, PSU)允许一组互不信任的参与方在不泄露除并集外任何额外信息的前提下,协同计算各自私有集合的并集。PSU 在现实世界中拥有广泛的应用场景,包括网络安全风险评估 \cite{LV04}、IP 黑名单与漏洞数据聚合 \cite{HLSSSYY16}、联合图计算 \cite{BllS05}、分布式网络监控 \cite{KS-CRYPTO-2005},以及作为构建支持全连接的隐私数据库 \cite{KRTW-ASIACRYPT-2019} 和 Private ID 协议 \cite{GMRSS-PKC-2021} 的核心组件。例如,在网络安全防御中,不同的安全机构往往希望联合各自的 IP 黑名单以获得更全面的威胁情报视图,但受限于数据的敏感性和隐私法规,直接共享黑名单是不可行的,PSU 为此类需求提供了理想的解决方案。

与受到广泛关注的隐私集合求交(Private Set Intersection, PSI),特别是技术已经高度成熟的两方 PSI (现有最高效的两方 PSI 协议 \cite{RR-CCS-2022} 的性能已足以媲美不安全的朴素哈希方案)相比,PSU 的研究起步较晚且发展相对缓慢。早期的两方 PSU 协议 \cite{KS-CRYPTO-2005} 及其后续工作主要依赖于加法同态加密(AHE)或复杂的通用电路,导致效率低下。直到 2019 年,Kolesnikov 等人 \cite{KRTW-ASIACRYPT-2019} 基于 OT 和对称密钥操作提出了首个适用于大规模数据集的高效的 两方 PSU 协议,实现了三个数量级的速度提升。随后的研究 \cite{GMRSS-PKC-2021, Jia-USENIX-2022} 继续在基于对称密钥的技术路线上探索,进一步降低了两方 PSU 协议的开销。张聪等人 \cite{ZCLZL-USENIX-2023} 和陈宇等人 \cite{ZCZDL-2024} 相继实现了线性复杂度,使得两方 PSU 技术也逐渐迈向成熟。

然而,多方隐私集合求并(MPSU)领域的研究进展依然十分缓慢,协议设计仍面临巨大挑战。根据底层使用的核心技术,现有的 MPSU 协议主要可以分为以下两类\footnote{本文不考虑基于通用 MPC 技术设计的MPSU方案,因其性能通常无法满足大规模应用需求。}:

\begin{itemize}
    \item \textbf{基于公钥密码技术的 MPSU:} 这一类协议主要依赖于公钥加密技术(如 AHE、ElGamal 等),代表性工作包括 \cite{KS-CRYPTO-2005, Frikken-ACNS-2007, VCE22, GNT-eprint-2023}。这类方案的一个共同缺陷是每个参与方都需要执行大量的公钥操作(如模幂运算),导致计算开销巨大,实际运行效率较低,难以支撑大规模数据的实时处理。
    
    \item \textbf{基于对称密钥技术的 MPSU:} 这一类协议主要依赖于对称密钥操作和不经意传输(OT)技术。截至目前,Liu 和 Gao \cite{LG-ASIACRYPT-2023} 提出的协议是该类别中唯一且最先进的工作。该协议在具体性能上远超所有基于公钥的方案(例如在特定设置下比 \cite{VCE22} 快两个数量级)。然而,该协议存在一个严重的安全性缺陷:它无法在标准半诚实模型下证明安全,而是依赖于一个较弱的“非共谋假设”(Non-collusion Assumption),即假设获得结果的参与方(Leader)不会与其他参与方进行共谋。在现实世界的零信任环境中,这一假设往往难以成立。
\end{itemize}

综上所述,目前 MPSU 领域存在一个重要的开放性问题:\textbf{能否在标准半诚实模型下基于 OT 和对称密钥操作构造一个安全高效的 MPSU 协议?}

本章将通过提出两个新的密码学组件 —— 批量秘密分享隐私成员测试(batch secret-shared private membership test, batch ssPMT)方秘密分享随机不经意传输(multi-party secret-shared ROT, mss-ROT),并基于这两个组件分别突破现有基于对称密钥的 MPSU 方案的性能瓶颈和消除其对非共谋谋假设的依赖,从而实现在标准半诚实模型下首个安全高效的基于对称密钥的 MPSU 协议,最终对这一问题给出了肯定的回答。

\section{技术路线概述}
\esection{Technical Overview}

\section{批量秘密分享隐私成员测试}
\esection{Batch Secret-Shared Private Membership Test}

\section{协议构造及安全性证明}
\esection{Protocol Construction and Security Proof}

\section{复杂度分析与对比}
\esection{Complexity Analysis and Comparison}

% \section{实现与性能分析}
% \esection{Implementation and Performance Analysis}

\section{本章小结}
\esection{Summary}