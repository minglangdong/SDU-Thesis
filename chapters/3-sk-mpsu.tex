\chapter{基于对称密钥的高效多方隐私集合求并协议}
\echapter{Efficient Multi-Party Private Set Union Protocol Based on Symmetric-Key Operations}

\section{引言}
\esection{Introduction}

隐私集合求并(Private Set Union, PSU)允许一组互不信任的参与方在不泄露除并集外任何额外信息的前提下,协同计算各自私有集合的并集。PSU 及其变体在现实世界中拥有广泛的应用场景,包括网络安全风险评估 \cite{LV04}、IP 黑名单与漏洞数据聚合 \cite{HLSSSYY16}、联合图计算 \cite{BllS05}、分布式网络监控 \cite{KS-CRYPTO-2005},以及作为构建支持全连接(Full Join)的隐私数据库 \cite{KRTW-ASIACRYPT-2019} 和 Private ID 协议 \cite{GMRSS-PKC-2021} 的核心组件。例如,在网络安全防御中,不同的安全机构往往希望联合各自的 IP 黑名单以获得更全面的威胁情报视图,但受限于数据的敏感性和隐私法规,直接共享黑名单是不可行的,PSU 为此类需求提供了理想的解决方案。

根据参与方的数量,隐私集合运算可分为两方和多方两类设置。在过去十年中,两方隐私集合求交(PSI)得到了广泛关注,技术已高度成熟,现有最高效协议 \cite{RR-CCS-2022} 的性能已足以媲美不安全的朴素哈希方案。相比之下,隐私集合求并(PSU)的研究起步较晚。早期的两方 PSU 协议 \cite{KS-CRYPTO-2005} 及其后续工作主要依赖于加法同态加密(AHE)或复杂的通用电路,导致效率低下。直到 2019 年,Kolesnikov 等人 \cite{KRTW-ASIACRYPT-2019} 提出了首个适用于大规模数据集的基于不经意传输(OT)的 PSU 协议,实现了三个数量级的速度提升。随后的研究 \cite{GMRSS-PKC-2021, Jia-USENIX-2022} 进一步降低了开销。特别是最近,文献 \cite{ZCLZL-USENIX-2023, ZCZDL-2024} 中的工作成功实现了计算和通信复杂度的线性化,使得两方 PSU 技术也逐渐迈向成熟。

然而,多方隐私集合求并(MPSU)领域的研究进展依然十分缓慢。尽管多方 PSI(MPSI)受益于两方技术的发展已涌现出许多高效构造 \cite{KMPRT-CCS-2017, NTY-CCS-2021, BNOP-AsiaCCS-2022},但 MPSU 协议的设计仍面临巨大挑战。根据底层使用的核心技术,现有的 MPSU 协议主要可以分为以下两类\footnote{此处仅考虑专用的 MPSU 解决方案,排除基于通用多方计算(MPC)电路的方案,因其性能通常无法满足大规模应用需求。}:

\begin{itemize}
    \item \textbf{基于公钥密码技术的 MPSU(PK-MPSU):} 这一类协议主要依赖于公钥加密技术(如 AHE、ElGamal 等),代表性工作包括 \cite{KS-CRYPTO-2005, Frikken-ACNS-2007, VCE22, GNT-eprint-2023}。这类方案的一个共同缺陷是每个参与方都需要执行大量的公钥操作(如模幂运算),导致计算开销巨大,实际运行效率较低,难以支撑大规模数据的实时处理。
    
    \item \textbf{基于对称密钥技术的 MPSU(SK-MPSU):} 这一类协议主要依赖于对称密钥操作和不经意传输(OT)技术。截至目前,Liu 和 Gao \cite{LG-ASIACRYPT-2023} 提出的协议是该类别中唯一且最先进的工作。该协议在具体性能上远超所有基于公钥的方案(例如在特定设置下比 \cite{VCE22} 快两个数量级)。然而,该协议存在一个严重的安全性缺陷:它无法在标准半诚实模型下证明安全,而是依赖于一个较弱的“非共谋假设”(Non-collusion Assumption),即假设获得结果的参与方(Leader)不会与其他参与方进行共谋。在现实世界的零信任环境中,这一假设往往难以成立。
\end{itemize}

综上所述,目前 MPSU 领域存在一个显著的开放性问题:\textbf{能否构造一个基于不经意传输和对称密钥操作的 MPSU 协议,在保持高效性能的同时,消除对非共谋假设的依赖,从而实现标准半诚实模型下的安全性?}

本章将通过提出一种新的核心原语及相应的协议构造,对这一问题给出肯定的回答。

\section{技术路线概述}
\esection{Technical Overview}

\section{批量秘密分享隐私成员测试}
\esection{Batch Secret-Shared Private Membership Test}

\section{协议构造及安全性证明}
\esection{Protocol Construction and Security Proof}

\section{复杂度分析与对比}
\esection{Complexity Analysis and Comparison}

% \section{实现与性能分析}
% \esection{Implementation and Performance Analysis}

\section{本章小结}
\esection{Summary}